\section{Exam 2014-2015}

\subsection{Question 1}
\subsubsection{Q1a}
An example of infinite dimensional vector space over a field is $\R[x]$, the set of polynomials with coefficients in $\R$.

\subsubsection{Q1b}
An vector space with exactly 16 elements is $\Z/16\Z=\{0,1,2,\ldots,15\}$
\subsubsection{Q1c}

\subsection{Question 2}
\subsubsection{Q2a}
To show that $\mathcal{B}$ forms a bsis, consider the matrix that represents $\mathcal{B}$
\begin{align*}
B=&\ 	\begin{pmatrix}
1  &1 & 1\\
 0 &1 &1 \\
  0& 0&1 
\end{pmatrix}\\
\st{using gaussian elimination we find that }
rref(B)=&\ \begin{pmatrix}
1  &0 & 0\\
 0 &1 &0 \\
  0& 0&1 
\end{pmatrix}
\st{or that $\dim{B}=3$ so $\mathcal{B}$ spans $V=\R^3$}
\end{align*}

\subsubsection{Q2b}
\paragraph{i)} 
\begin{align*}
	\st{Denote the equivalence class $[v]$ for $v\in V$ by }
	[v]= \{v+u:u\in U\}
	\st{and addition and multiplication is defined as follows}
k[n]=&\ [kn]
\st{for all $k\in \R$, and}
[v_1]+[v_2]=&\ [v_1+v_2]	
\st{thus the canonical mapping is simply}
can(v):V\to V/U =&\ [v]\\
\st{and therefore $\ker{(can)}={0}$ as }
[0] =&\ \{0+u:u\in U\}
\end{align*}



\subsection{Question 3}

\subsection{Question 4}

\subsection{Question 5}