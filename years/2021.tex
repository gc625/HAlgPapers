\section{Exam 2020-2021}

\subsection{Question 1}
\subsubsection{Q1a}
\textbf{F:} A \textbf{noncommutative ring} is a ring such that 
\begin{align*}
	a\cdot b \neq&\ b\cdot a 
\end{align*}
and an element in a ring is a \textbf{zero divisor} if there exists non-zero $b$ such that 
\begin{align*}
	ab=0 \quad \text{or}&\quad ba=0
\end{align*}
An example of this is $\mathbb{H}$: The ring of quaternions. \href{https://mathsci2.appstate.edu/~sjg/class/3110/mathfestalg2000/quaternions.html#:~:text=abelian\%20under\%20multiplication.-,Quaternions\%20are\%20Not\%20an\%20Integral,but\%20have\%20no\%20zero\%2Ddivisors.}{proof if interested}
\subsubsection{Q1b}
\textbf{F:} Example: $V=\R$, $W=\R$
\begin{align*}
f(x)=&\ \binom{x}{0}	\\
g\binom{x}{y}=&\ 0	\\
\end{align*}
Since $g\circ f(x)=x$ but $\dim{V}\neq\dim{W}$


\subsubsection{Q1c}
Question can be rephrased as: Do $n\times n$ matrices with odd $n$ always have (a real) eigenvalue?\\
\noindent\textbf{T:} because the characteristic polynomial will have an odd degree, by the intermediate value theorem (\href{https://math.stackexchange.com/questions/689575/proof-that-every-polynomial-of-odd-degree-has-one-real-root}{proof}), it must have at least one real root $\implies$ matrix has at one real eigenvalue.
\subsubsection{Q1d}

\textbf{F:}



\subsection{Question 2}
\subsection{Question 3}