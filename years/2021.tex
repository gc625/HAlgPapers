\section{Exam 2020-2021}

\subsection{Question 1}
\subsubsection{Q1a}
\textbf{F:} A \textbf{noncommutative ring} is a ring such that
\begin{align*}
	a\cdot b \neq & \ b\cdot a
\end{align*}
and an element in a ring is a \textbf{zero divisor} if there exists non-zero $b$ such that
\begin{align*}
	ab=0 \quad \text{or} & \quad ba=0
\end{align*}
An example of this is $\mathbb{H}$: The ring of quaternions. \href{https://mathsci2.appstate.edu/~sjg/class/3110/mathfestalg2000/quaternions.html#:~:text=abelian\%20under\%20multiplication.-,Quaternions\%20are\%20Not\%20an\%20Integral,but\%20have\%20no\%20zero\%2Ddivisors.}{proof if interested}
\subsubsection{Q1b}
\textbf{F:} Example: $V=\R$, $W=\R$
\begin{align*}
	f(x)=          & \ \binom{x}{0} \\
	g\binom{x}{y}= & \ 0            \\
\end{align*}
Since $g\circ f(x)=x$ but $\dim{V}\neq\dim{W}$


\subsubsection{Q1c}
Question can be rephrased as: Do $n\times n$ matrices with odd $n$ always have (a real) eigenvalue?\\
\noindent\textbf{T:} because the characteristic polynomial will have an odd degree, by the intermediate value theorem (\href{https://math.stackexchange.com/questions/689575/proof-that-every-polynomial-of-odd-degree-has-one-real-root}{proof}), it must have at least one real root $\implies$ matrix has at one real eigenvalue.
\subsubsection{Q1d}

\textbf{F:} \textbf{The group of units} $R^\times$ in a ring $R$ is the set of elements $a$ with multiplicative inverse in $R$, or
\begin{align*}
	R^\times =\{a\in R: \exists\ a^{-1}\in R \text{ s.t. } aa^{-1}=a^{-1}a=1 \}
\end{align*}
\textbf{Cyclic group} is a group that can be generated by one element.\\
\textbf{Note:} $(\Z/m\Z)^\times$ are elements in the ring that are relatively prime to $m$, i.e.: $(\Z/m\Z)^\times=\{a\in(\Z/m\Z):gcd(a,m)=1\}$\\
\noindent\textbf{Note 2:} only for prime $m$ is $(\Z/m\Z)^\times$ cyclic.
\bigbreak
\noindent
Consider $(\Z/8\Z)^\times=\{1,3,5,7\}$, the group cannot be generated by one element.
\subsubsection{Q1e}
\textbf{T:} Since $A^T$ and $A$ have the same characteristic polynomial $\implies$ same eigenvalues.
\newpage
\subsubsection{Q1f}
\textbf{T:} $R=\R[x]$: polynomials with real coefficients\\ $_R<x^3+3x+7>$: think of it as resulting set of polynomials after multiplying every polynomial in $R$ by $x^3+3x+7$. More precisely:
\begin{align*}
	_R<x^3+3x+7>= & \ \{(x^3+3x+7)\cdot a: a\in R\}
\end{align*}
The \textbf{Quotient (or Factor) Ring} $R/ I$ is then the cosets of $I$ in $R$ subject to special addition and multiplication\footnote{See THEOREM 3.6.4, pg 53}
And the \textbf{equivalence relation} for cosets is defined to be
$$x\sim y \iff x-y\in I$$ for $x,y\in R$

Since,
\begin{align*}
	((x^2 + 1) + I) ((2x^2 + 3x) + I) = & \ ((2x^4+3x^4+2x^2+3x)+I)
	\st{let}
	A =                                 & \ 2x^4+3x^4+2x^2+3x       \\
	\st{and}
	B =                                 & \ -4x^2 - 20x - 21        \\
	A- B =                              & \ 2x^4+3x^3+6x^2+23x+21   \\
	\st{since}
	\frac{A-B}{x^3+3x+7}=               & \ 2x+3                    \\
	\st{we have $A-B\in I$ and hence cosets $(A+I)$ and $(B+I)$ are equivalent}
\end{align*}


\subsubsection{Q1g}
\textbf{F:} An inner product must have the following properties:
\begin{align*}
	(\lambda \vec{x}+\mu\vec{y},\vec{z}) = & \ \lambda(\vec{x},\vec{z} )+\mu(\vec{y},\vec{z}) \tag{1} \\
	(\vec{x},\vec{y})=                     & \ {(\vec{y},\vec{x})}\tag{2}                             \\
	(\vec{x},\vec{x})\geq                  & \ 0 \tag{3}
\end{align*}
The proposed inner product violates property 3. Consider the polynomial
\begin{align*}
	P(x)=                         & \ \prod_{i=1}^{n-1}(x-i) \\
	\st{then}
	\textcolor{red}{(P(x),P(x))=} & \ \textcolor{red}{0}
	\st{\textcolor{red}{I dont see how it equals 0, maybe im misreading sth}}
\end{align*}

\subsubsection{Q1h}

\textbf{T:} The complex inner product has the following properties:
\begin{align*}
	(\lambda \vec{x}+\mu\vec{y},\vec{z}) = & \ \lambda(\vec{x},\vec{z} )+\mu(\vec{y},\vec{z}) \tag{1} \\
	(\vec{x},\vec{y})=                     & \ \overline{(\vec{y},\vec{x})}\tag{2}                    \\
	(\vec{x},\vec{x})\geq                  & \ 0 \tag{3}
\end{align*}
Fairly straightforward by checking it satisfies the three properties. \textcolor{red}{I think you can probably claim properties 1 trivial/clear from definition}. Proving 2:
\begin{align*}
	\left(\binom{x_1}{x_2},\binom{y_1}{y_2}\right)= & \ 4x_1\overline{y_1}-2x_1\overline{y_2}-2x_2\overline{y_1}+3x_2\overline{y_2} \\
	=                                               & \ (2x_1-x_2)\overline{(2y_1-y_2)}+2x_2\overline{y_2}
	\st{proving property 3}
	\left(\binom{x_1}{x_2},\binom{x_1}{x_2}\right)= & \ |2x_1-x_2|^2+2|x_2|^2
\end{align*}

\subsubsection{Q1i}
\textbf{F:} The \textbf{Image} $\Ima f$ of a linear map $f:V\to W$  is $f(V)\in W$. (Everything in $W$ that can be mapped to by $f$).\\
The \textbf{kernel} $\ker f$ is the set $\{v\in V: f(v)=0\}$ (everything in $V$ that is mapped to $0_W$)

Consider the example $A=\begin{pmatrix}
		0 & 1 \\
		0 & 0
	\end{pmatrix}$. The vector $v=\dbinom{1}{0}$ is in both the image and kernel. Since

\begin{align*}
	\begin{pmatrix}
		0 & 1 \\
		0 & 0
	\end{pmatrix}\binom{1}{0}= & \ \binom{0}{0} \quad\text{(in kernel)} \\
	\begin{pmatrix}
		0 & 1 \\
		0 & 0
	\end{pmatrix}\binom{0}{1}= & \ \binom{1}{0}\quad\text{(in image)}   \\
\end{align*}

\subsubsection{Q1j}
\textcolor{red}{someone else explain it better pls}


\subsection{Question 2}
\subsubsection{Q2a}
$m_A(x)$ for $A=\begin{pmatrix}
		\lambda&0\\ 0& \lambda\end{pmatrix}$ is simply $x-\lambda$ because all the $\lambda$'s lie on the diagonal. so subtracting $\lambda I$ from $A$ would equal 0.\\
$m_A(x)$ for $A=\begin{pmatrix}
		\lambda&1\\ 0& \lambda\end{pmatrix}$, not exactly sure how besides calculating characteristic polynomial.
\begin{align*}
	\chi_A(x)= & \ \det(A-xI)                                   \\
	=          & \ \det \begin{pmatrix}
		                    \lambda-x&1\\ 0& \lambda-x\end{pmatrix} \\
	=          & \ (\lambda-x)^2
\end{align*}


\subsubsection{Q2b}
A subset $I$ of a ring $R$ is an ideal if
\begin{enumerate}
	\item $I\neq \emptyset$
	\item $I$ is closed under subtraction
	\item for all $i \in I$ and $r \in R$ we have $ri, ir \in I$.
\end{enumerate}
Clearly $m_A\in I_A$ since $\cdot m_A(A)= 0 $. So 1 is true.
\begin{align*}
	\st{For any}
	q(x)\in       & \  F[x]
	\st{it follows that}
	q(x)m_A(x)\in & \ I_A
	\st{since}
	q(A)m_A(A) =  & \ q(A)\cdot 0 \\
	=             & \ 0
\end{align*}
\textcolor{red}{solutions claims it is closed under addition, not sure if the difference is meaningful, but we can prove that it is closed until subtraction via the following:}
\begin{align*}
	\st{Consider $p,q\in F[x]$, clearly}
	p(x)m_A(x)\in        & \ I_A        \\
	\st{and}
	q(x)m_A(x)\in        & \ I_A        \\
	\st{Since}
	p(x)-q(x) \in        & \ F[x]       \\
	\st{and}
	(p(A)-q(A))m_A(A) =  & (p(A)-q(A))0 \\
	=                    & 0            \\
	\st{so }
	(p(x)-q(x))m_A(x)\in & \ I_A
	\st{and therefore $I_A$ is closed under subtraction}
\end{align*}
\subsection{Question 3}