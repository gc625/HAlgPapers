\section{Exam 2019-2020}

\subsection{Question 1}
\subsection{Question 2}
\subsection{Question 3}
\subsubsection{Q3a}
\paragraph{(i)} Recall that the complex inner product has the following properties:
\begin{align*}
(\lambda \vec{x}+\mu\vec{y},\vec{z}) =&\ \lambda(\vec{x},\vec{z} )+\mu(\vec{y},\vec{z}) \tag{1}\\
(\vec{x},\vec{y})=&\ \overline{(\vec{y},\vec{x})}\tag{2}\\
(\vec{x},\vec{x})\geq &\ 0 \tag{3}
\st{and $T$ being self adjoint}
(T\vec{x},{y})=&\ (\vec{x},T\vec{y})
\st{Thus, let $v$ be an eigenvector with eigenvalue $\lambda$.}
\lambda (x,x) =&\ (x,\lambda x) \quad\textit{(property 1)}\\
=&\ (x,T x) \quad\textit{(eigenvec property)}\\
=&\ (T x,x) \quad\textit{(self adjoint prop)}\\
=&\ (\lambda x, x) \quad\textit{(eigenvec propety)}\\
=&\ \overline{(x,\lambda x)} \quad\textit{(property 2)}\\
=&\ (x,\overline{\lambda}x) \quad\textit{\textcolor{red}{(not sure why we can claim $x=\bar{x}$ here)}}\\
=&\ \overline{\lambda}(x,x)\quad\textit{(property 1)}
\st{implying that $\lambda=\overline{\lambda}$, which can only be true is $\lambda\in\R$.}
\end{align*}

\paragraph{(ii)} 
The proof in (i) fails if $T$ is not self adjoint due to the fact that we cannot claim $(x,Tx)=(Tx,x)$ unless $T$ is self adjoint. 
\paragraph{(iii)}

\subsection{Question 4}