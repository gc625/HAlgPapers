\section{Exam 2019-2020}

\subsection{Question 1}
\subsection{Question 2}
\subsubsection{Q2a}
Here provides a faster method:\\
\begin{alignat*}{2}
	f(1, 1, 1) & = (1, 2)   & = 0 (2, 1) + 1 (1, 2) \\
	f(1, 2, 1) & = (-1, -2) & = 0 (2, 1) - 1 (1, 2) \\
	f(0, 1, 2) & = (4, -4)  & = 4 (2, 1) - 4 (1, 2)
\end{alignat*}
Hence,
\begin{alignat*}{2}
	{}_\mathcal{B}[f]_{\mathcal{A}} & =
	\begin{pmatrix}
		0 & 0  & 4  \\
		1 & -1 & -4
	\end{pmatrix}
\end{alignat*}
Standard method:\\
To get the representing matrix of $f$, evaluate $f$ at the standard basis.
\begin{align*}
	f(1,0,0)= & \ (0,6)                                                          \\
	f(0,1,0)= & \ (-2,-4)                                                        \\
	f(0,0,1)= & \ (3,0)                                                          \\
	\st{Lets call the representing matrix $M$}
	M=        & \ \begin{pmatrix}
		              0 & -2 & 3 \\
		              6 & -4 & 0
	              \end{pmatrix}
	\st{Note that $f$ currently is mapping from $\R^3$ to $\R^2$ w.r.t. to standard bases $S(3)$ and $S(2)$. So $MA$, where $A$ is matrix formed by the ordered basis $\mathcal{A}$}
	A=        & \ \begin{pmatrix}
		              1 & 1 & 0 \\
		              1 & 2 & 1 \\
		              1 & 1 & 2 \\
	              \end{pmatrix}                                                 \\
	MA=       & \ \begin{pmatrix}
		              0 & -2 & 3 \\
		              6 & -4 & 0
	              \end{pmatrix}\begin{pmatrix}
		                           1 & 1 & 0 \\
		                           1 & 2 & 1 \\
		                           1 & 1 & 2 \\
	                           \end{pmatrix}                                    \\
	=         & \ \begin{pmatrix}1&-1&4\\ 2&-2&-4\end{pmatrix}
	\st{represents $_{S(2)}[f]_{\mathcal{A}}$.}
	B= \begin{pmatrix}
		   2 & 1 \\
		   1 & 2 \\
	   \end{pmatrix}
	\st{finally $_{\mathcal{B}}[f]_{\mathcal{A}}$ is then $B^{-1}MA$}
	=         & \ \begin{pmatrix}
		              2 & 1 \\
		              1 & 2 \\
	              \end{pmatrix}^{-1}\begin{pmatrix}1&-1&4\\ 2&-2&-4\end{pmatrix} \\
	=         & \ \begin{pmatrix}0&0&4\\ 1&-1&-4\end{pmatrix}
\end{align*}

\subsubsection{Q2b}
\paragraph{(i)}
Consider $A=\begin{pmatrix}
		a & b \\
		c & d\end{pmatrix}$
\begin{align*}
	Tr(A)^2=     & \ (a+d)^2       \\
	Tr(A^2)=     & \ a^2+bc+bc+d^2 \\
	\st{by construction, we have}
	Tr(A)^2=     & \ Tr(A^2)       \\
	(a+d)^2=     & \ a^2+2bc+d^2   \\
	a^2+2ad+d^2= & \ a^2+2bc+d^2   \\
	\st{which implies}
	ad=          & \ bc
	\st{since $\det{A}=ad-bc$, it follows that $\det{A}=0$ }
\end{align*}
Alternative method:\\
By Cayley-Hamilton theorem,
\begin{equation*}
	A^2 - \operatorname{Tr}{A} + \operatorname{det}{A} \mathbb{I}_2 = 0
\end{equation*}
Take trace on both side:
\begin{equation*}
	\begin{aligned}
		\operatorname{Tr}{A^2} - (\operatorname{Tr}{A})^2 + 2 \operatorname{det}{A} & = 0 \\
		2 \operatorname{det}{A}                                                     & = 0
	\end{aligned}
\end{equation*}
Since \(2\) is a unit in \(\mathbb{C}\), so \(\operatorname{det}{A} = 0.\)
\paragraph{(ii)}
No.
Notice in the previous proof requires a condition is 2 is a unit.
So let \(F = \mathbb{F}_2\), then let \(A = \mathbb{I}_2 = \begin{pmatrix}1&0\\0&1\end{pmatrix}\). \(\Tr{A^2} = (\Tr{A})^2\) but \(\det{A} = 1 \neq 0\)
\subsubsection{Q2c}
A mapping is a ring homomorphism if
\begin{align*}
	f(x+y)= & \ f(x)+f(y) \\
	f(xy)=  & \ f(x)f(y)
\end{align*}
\paragraph{(i)}
\begin{align*}
	\st{$f_1$ violates the first property, consider}
	f(0)=       & \ \begin{pmatrix}
		                0 & 0 \\
		                0 & 0 \\
	                \end{pmatrix}
	\st{but}
	f(1)+f(-1)= & \ \begin{pmatrix}
		                1 & 0  \\
		                0 & -1 \\
	                \end{pmatrix}\begin{pmatrix}
		                             -1 & 0 \\
		                             0  & 1 \\
	                             \end{pmatrix} \\
	=           & \  \begin{pmatrix}
		                 -1 & 0  \\
		                 0  & -1 \\
	                 \end{pmatrix}
	\st{so}
	f(0)\neq    & \ f(-1)+f(1)
\end{align*}
\paragraph{(ii)}
\(f_2\) violates the first property,
consider \(p_1(X) = X + 1\) and \(p_2(X) = -X\).
\begin{align*}
	f_2(p_1(X))               & = 1                       \\
	f_2(p_2(X))               & = -1                      \\
	f_2(p_1(X) + p_2(X))      & = f_2(1) = 1              \\
	f_2(p_1(X)) + f_2(p_2(X)) & \neq f_2(p_1(X) + p_2(X))
\end{align*}
\subsubsection{Q2d}
\paragraph{(i)} \textbf{Yes}. See Exercise 80.
\paragraph{(ii)} \textbf{No}. Let \(P = Q\) be constant and it violates property 3.
\paragraph{(iii)} \textbf{No}. Let \(P = \prod_{i=1}^{n}{(X - x_i)}\) and it violates property 3.
\subsection{Question 3}
\subsubsection{Q3a}
See THEOREM 5.3.7.
\paragraph{(i)} Recall that the complex inner product has the following properties:
\begin{align*}
	(\lambda \vec{x}+\mu\vec{y},\vec{z}) = & \ \lambda(\vec{x},\vec{z} )+\mu(\vec{y},\vec{z}) \tag{1} \\
	(\vec{x},\vec{y})=                     & \ \overline{(\vec{y},\vec{x})}\tag{2}                    \\
	(\vec{x},\vec{x})\geq                  & \ 0 \tag{3}
	\st{and $T$ being self adjoint}
	(T\vec{x},{y})=                        & \ (\vec{x},T\vec{y})
\end{align*}
Thus, let \(v\) be an eigenvector with eigenvalue \(lambda\).
\begin{align*}
	\lambda(\vec{v}, \vec{v})
	 & = (\lambda\vec{v}, \vec{v}) \quad\textit{(property 1)}                       \\
	 & = (T\vec{v}, \vec{v}) \quad\textit{(eigenvector property)}                   \\
	 & = (\vec{v}, T\vec{v}) \quad\textit{(self adjoint property)}                  \\
	 & = (\vec{v}, \lambda\vec{v}) \quad\textit{(eigenvector property)}             \\
	 & = \overline{(\lambda\vec{v}, \vec{v})} \quad\textit{(property 2)}            \\
	 & = \overline{\lambda}\overline{(\vec{v}, \vec{v})} \quad\textit{(property 1)} \\
	 & = \overline{\lambda}(\vec{v}, \vec{v}) \quad\textit{(property 2)}            \\
	\lambda(\vec{v}, \vec{v}) = \overline{\lambda}(\vec{v}, \vec{v})
	 & \implies \lambda = \overline{\lambda}                                        \\
	 & \implies \lambda \in \R
\end{align*}
\paragraph{(ii)}
The proof in (i) fails if $T$ is not self adjoint due to the fact that we cannot claim $(x,Tx)=(Tx,x)$ unless $T$ is self adjoint.
\paragraph{(iii)}
Both \(\lambda\) and \(\mu\) are real by part (i).
\begin{align*}
	\lambda(\vec{v}, \vec{w})
	 & = (T\vec{v}, \vec{w})   \\
	 & = (\vec{v}, T\vec{w})   \\
	 & = \mu(\vec{v}, \vec{w}) \\
	\st{which cannot be true unless $(v,w)=0$ as $\lambda$ and $\mu$ are distinct eigenvalues.}
\end{align*}
\paragraph{(iv)}
By constructive method,
\begin{align*}
	\lambda            & = 1                 \\
	\mu                & = 2                 \\
	\lambda            & \ne \mu             \\
	\vec{v}            & = \begin{pmatrix}
		                       1 \\
		                       0
	                       \end{pmatrix}    \\
	\vec{w}            & = \begin{pmatrix}
		                       1 \\
		                       1
	                       \end{pmatrix}    \\
	(\vec{v}, \vec{w}) & \ne 0               \\
	D                  & = \begin{pmatrix}
		                       \lambda & 0   \\
		                       0       & \mu
	                       \end{pmatrix}    \\
	                   & = \begin{pmatrix}
		                       1 & 0 \\
		                       0 & 2
	                       \end{pmatrix}    \\
	P                  & = \begin{pmatrix}
		                       \vec{v} & \vec{w}
	                       \end{pmatrix} \\
	                   & = \begin{pmatrix}
		                       1 & 1 \\
		                       0 & 1
	                       \end{pmatrix}    \\
	T                  & = PDP^{-1}          \\
	                   & = \begin{pmatrix}
		                       1 & 1 \\
		                       0 & 2
	                       \end{pmatrix}
\end{align*}
\subsubsection{Q3b}
\paragraph{(i)}
\paragraph{(ii)}
\paragraph{(iii)}
\paragraph{(iv)}
\subsection{Question 4}
\subsubsection{Q4a}
\begin{align*}
	\sum_{i = 0}^{n}{p_{i}(\sqrt{\alpha})^i}
	 & = \sum_{i = 0}^{\lfloor \frac{n}{2} \rfloor}{p_{2i}(\sqrt{\alpha})^{2i}}
	+ \sum_{i = 0}^{\lfloor \frac{n}{2} \rfloor}{p_{2i+1}(\sqrt{\alpha})^{2i+1}} \\
	 & = \sum_{i = 0}^{\lfloor \frac{n}{2} \rfloor}{p_{2i}\alpha^i}
	+ \sum_{i = 0}^{\lfloor \frac{n}{2} \rfloor}{p_{2i+1}\alpha^i\sqrt{\alpha}}  \\
	 & = \left(\sum_{i = 0}^{\lfloor \frac{n}{2} \rfloor}{p_{2i}\alpha^i}\right)
	+ \left(\sum_{i = 0}^{\lfloor \frac{n}{2} \rfloor}{p_{2i+1}\alpha^i}\right)\sqrt{\alpha}
\end{align*}
Therefore, $f$ maps every rational polynomial to a real number of the form $a+b\sqrt{\alpha}$ for some $a,b\in\Q$
Now we show it is surjective by \(f(a+bX) = a + b\sqrt{\alpha}\).
\newline
Claim:
\begin{equation*}
	\ker{f} =
	\begin{cases}
		{}_{\Q[X]}\langle X - \sqrt{\alpha} \rangle & \sqrt{\alpha} \in \Q \\
		{}_{\Q[X]}\langle X^2 - \alpha \rangle      & \text{otherwise}
	\end{cases}
\end{equation*}
If \(\sqrt{\alpha} \in \Q\), \(p(X) = q(X) (X - \sqrt{\alpha}) + a\).
\begin{align*}
	p(X) \in \ker{f}
	 & \iff f(p(X)) = 0                                               \\
	 & \iff f(q(X) (X - \sqrt{\alpha}) + a) = 0                       \\
	 & \iff f(a) = 0                                                  \\
	 & \iff a = 0                                                     \\
	 & \iff (X - \sqrt{\alpha}) | p(X)                                \\
	 & \implies \ker{f} = {}_{\Q[X]}\langle X - \sqrt{\alpha} \rangle
\end{align*}
If \(\sqrt{\alpha} \notin \Q\), \(p(X) = q(X) (X^2 - \alpha) + a + bX\).
\begin{align*}
	p(X) \in \ker{f}
	 & \iff f(p(X)) = 0                                          \\
	 & \iff f(q(X) (X^2 - \alpha) + a + bX) = 0                  \\
	 & \iff f(a + bX) = 0                                        \\
	 & \iff a + b \sqrt{\alpha} = 0                              \\
	 & \iff a = b = 0 \quad\text{($\alpha$ is irrational)}       \\
	 & \iff (X^2 - \alpha) + a + bX) | p(X)                      \\
	 & \implies \ker{f} = {}_{\Q[X]}\langle X^2 - \alpha \rangle
\end{align*}
Hence \(\ker{f}\) is a principle ideal.
\subsubsection{Q4b}
Yes.
If \(\sqrt{\alpha}\) is rational, then \(\im{f} = \Q\) which is definitely a field.
If \(\sqrt{\alpha}\) is irrational, then using high school math,
\begin{align*}
	(a + b\sqrt{\alpha})^{-1} = \frac{a - b \sqrt{\alpha}}{a^2 - \alpha b^2}
\end{align*}
Other properties of field can be verified easily.
So \(\im{f}\) is a field.