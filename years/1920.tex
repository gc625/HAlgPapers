\section{Exam 2019-2020}

\subsection{Question 1}
\subsection{Question 2}
\subsubsection{Q2a}
To get the representing matrix of $f$, evaluate $f$ at the standard basis.
\begin{align*}
	f(1,0,0)=&\ (0,6)\\ 
	f(0,1,0)=&\ (-2,-4)\\ 
	f(0,0,1)=&\ (3,0)\\ 
\st{Lets call the representing matrix $M$}
M=&\ \begin{pmatrix}
0&-2&3\\
6&-4&0	
\end{pmatrix}
\st{Note that $f$ currently is mapping from $\R^3$ to $\R^2$ w.r.t. to standard bases $S(3)$ and $S(2)$. So $MA$, where $A$ is matrix formed by the ordered basis $\mathcal{A}$}
	A=&\ \begin{pmatrix}
		1&1&0\\
		1&2&1\\
		1&1&2\\
	\end{pmatrix}\\
	MA=&\ \begin{pmatrix}
0&-2&3\\
6&-4&0	
\end{pmatrix}\begin{pmatrix}
		1&1&0\\
		1&2&1\\
		1&1&2\\
	\end{pmatrix}\\
	=&\ \begin{pmatrix}1&-1&4\\ 2&-2&-4\end{pmatrix}
	\st{represents $_{S(2)}[f]_{\mathcal{A}}$.}
	B= \begin{pmatrix}
		2&1\\
		1&2\\
	\end{pmatrix}
\st{finally $_{\mathcal{B}}[f]_{\mathcal{A}}$ is then $B^{-1}MA$}
=&\ \begin{pmatrix}
		2&1\\
		1&2\\
	\end{pmatrix}^{-1}\begin{pmatrix}1&-1&4\\ 2&-2&-4\end{pmatrix}\\
	=&\ \begin{pmatrix}0&0&4\\ 1&-1&-4\end{pmatrix}
\end{align*}

\subsubsection{Q2b}
\paragraph{(i)}


Consider $A=\begin{pmatrix}
	a&b\\
	c&d\end{pmatrix}$
\begin{align*}
Tr(A)^2=&\ (a+d)^2\\
Tr(A^2)=&\ a^2+bc+bc+d^2\\
\st{by construction, we have}
Tr(A)^2=&\ Tr(A^2)\\
(a+d)^2=&\ a^2+2bc+d^2\\
a^2+2ad+d^2=&\ a^2+2bc+d^2\\
\st{which implies}
ad=&\ bc
\st{since $\det{A}=ad-bc$, it follows that $\det{A}=0$ }
\end{align*}

\paragraph{(ii)}


\textcolor{red}{idk}



\subsubsection{Q2c}
\paragraph{(i)}


A mapping is a ring homomorphism if 
\begin{align*}
	f(x+y)=&\ f(x)+f(y)\\
	f(xy)=&\ f(x)f(y)
\end{align*}
\begin{align*}
\st{$f_1$ violates the first property, consider}
f(0)=&\ \begin{pmatrix}
	0&0\\
	0&0\\
\end{pmatrix}
\st{but}
f(1)+f(-1)=&\ \begin{pmatrix}
	1&0\\
	0&-1\\
\end{pmatrix}\begin{pmatrix}
	-1&0\\
	0&1\\
\end{pmatrix}\\
=&\  \begin{pmatrix}
	-1&0\\
	0&-1\\
\end{pmatrix}
\st{so}
f(0)\neq&\ f(-1)+f(1)
\end{align*}
\paragraph{(ii)}
\textcolor{red}{lol cant think of one}



\subsection{Question 3}
\subsubsection{Q3a}
\paragraph{(i)} Recall that the complex inner product has the following properties:
\begin{align*}
(\lambda \vec{x}+\mu\vec{y},\vec{z}) =&\ \lambda(\vec{x},\vec{z} )+\mu(\vec{y},\vec{z}) \tag{1}\\
(\vec{x},\vec{y})=&\ \overline{(\vec{y},\vec{x})}\tag{2}\\
(\vec{x},\vec{x})\geq &\ 0 \tag{3}
\st{and $T$ being self adjoint}
(T\vec{x},{y})=&\ (\vec{x},T\vec{y})
\st{Thus, let $v$ be an eigenvector with eigenvalue $\lambda$.}
\lambda (x,x) =&\ (x,\lambda x) \quad\textit{(property 1)}\\
=&\ (x,T x) \quad\textit{(eigenvec property)}\\
=&\ (T x,x) \quad\textit{(self adjoint prop)}\\
=&\ (\lambda x, x) \quad\textit{(eigenvec propety)}\\
=&\ \overline{(x,\lambda x)} \quad\textit{(property 2)}\\
=&\ (x,\overline{\lambda}x) \quad\textit{\textcolor{red}{(not sure why we can claim $x=\bar{x}$ here)}}\\
=&\ \overline{\lambda}(x,x)\quad\textit{(property 1)}
\st{implying that $\lambda=\overline{\lambda}$, which can only be true is $\lambda\in\R$.}
\end{align*}

\paragraph{(ii)} 
The proof in (i) fails if $T$ is not self adjoint due to the fact that we cannot claim $(x,Tx)=(Tx,x)$ unless $T$ is self adjoint.
 
\paragraph{(iii)}
\begin{align*}
\st{By the self adjoint property}
(Tv,w)=&\ (v,Tw)	\\
\st{since $Tv=\lambda v$ and $Tw=\mu w$} 
(\lambda v , w) =&\ (v,\mu w)\\
\st{by property 1}
(\lambda v , w) =&\  \lambda(v,w)\\
(v,\mu w) =&\ \mu(v,w)\\
\st{thus}
\lambda(v,w)=&\ \mu(v,w)\\
\st{which cannot be true unless $(v,w)=0$ as $\lambda$ and $\mu$ are distinct eigenvalues.}
\end{align*}
\paragraph{(iv)}

\subsection{Question 4}

\subsubsection{Q4a}
$\sqrt{\alpha}\in\Q$ implies that $\alpha\in\Q$. Thus result of the mapping $f(p_nX^n+\cdots+p_1X+p_0)= p_n(\sqrt{\alpha})^n+\cdots+p_1\sqrt{\alpha}+p_0$ can be simplied into the form $a+b\sqrt{\alpha}$, where $a,b\in\Q$. If $n$ is even. 
\begin{align*}
p_n(\sqrt{\alpha})^n+\cdots+p_1\sqrt{\alpha}+p_0=&\ p_0+p_2\alpha+p_4\alpha^2+\cdots+p_n\alpha^{\frac{n}{2}}\\
&+p_1\sqrt{\alpha}+p_3\alpha\sqrt{a}+\cdots+p_{n-1}\alpha^{\frac{n}{2}-1}\sqrt{\alpha}\\
=&\ \sum_{i=0}^{n/2}p_{2i}\alpha^{i}+\sum_{i=0}^{\frac{n}{2}-1}p_{2i+1}\alpha^{i}\sqrt{\alpha} 
\st{let}
a=&\ \sum_{i=0}^{n/2}p_{2i}\alpha^{i}\\
\st{and}
b=&\ \sum_{i=0}^{\frac{n}{2}-1}p_{2i+1}\alpha^{i}
\st{to get}
\ p_0+p_2\alpha+p_4\alpha^2+\cdots+p_n\alpha^{\frac{n}{2}}=&\ a+b\sqrt{\alpha}
\st{proof is same if $n$ is odd except for an additional term. Therefore, $f$ maps every rational polynomial to a real number of the form $a+b\sqrt{\alpha}$ for some $a,b\in\Q$, hence}
\Ima{f}=&\ \{a+b\sqrt{\alpha}:a,b\in\Q\}
\end{align*} 



